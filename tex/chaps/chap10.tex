\chapter{Introduction} % Main chapter title
\label{intro} % For referencing the chapter elsewhere, use \ref{introduction} 


\section{Theory and heuristics of life-cycle investments}
One of the most important investment decisions, individuals face in their lives, is investment in a retirement portfolio. Historically, retirement funds paid out fixed annuities to all of their clients. However, in 1952, with the rise of post-war instability, Teachers Insurance and Annuity Association (TIAA) allowed households to diversify their retirement portfolios for the first time. Academic economists, who were among the clients of TIAA, have reacted by allocating 50\% of their funds to riskless bonds and 50\% to risky stocks. Performing a deeper analysis, \citet{markowitz} has created a Modern Portfolio Theory and founded a field of financial economics.
 
\paragraph*{} 
With the development of the field, life-cycle investment strategies, which proposed adjusting portfolios at every age, gained recognition. Many funds offered "investment menus" --- predetermined portfolio allocations for every age, to help their clients make optimal decisions without dealing with the complex theory. However, Turkish consulting firms have not included easy-to-comprehend lifecycle strategies (investment menus) in their bulletins, and didn't try to spread transparent information. We will fill this important gap in our paper, but firstly we will review the state of Turkish pension system.

\section{Turkish pension system}
The main pension funds in Turkey have been public for a long time: three main options existed: SSK for public and private sector workers, ES for civil servants, and Bag-Kur for self-employed workers and farmers. In 2006 they all merged into SGK. Private pensions have gained pace recently. As of January, 2017 a new clause of Turkish Labor Law came into action, that automatically enrolled every wage earner younger than 45 years into "Individual Retirement Scheme" --- a private pension fund. To further incentivize people not to opt out, the government promised to subsidize 25\% of their monthly contributions (as long as this wouldn't exceed 25\% of minimal wage). According to PwC research, this has tremendously increased fund sizes. Under the current system individuals may retire after contributing to a pension fund for at least 10 years and at least reaching the age of 56.
\paragraph*{}
The largest retirement funds in Turkey are listed in the Table \ref{table:emekli}. All of them offer 3-4 default investment options with varying degrees of riskiness but they are not lifecycle investment strategies mentioned above. They also provide flexible investment options with ability to change portfolio allocation up to six times a year, but they are not very popular as they assume active involvement in the portfolio and require a certain level of financial literacy. Not much academic research has been done on Turkish pension systems, and the existing research doesn't provide easy solutions. A recent example of this is Iscanoglu-Cekic's (2016) paper which doesn't consider life cycles and uses dynamic programming in the solution, which is also not accessible to wide audience.

\section{Focus of this thesis}
In this thesis we will consider the general framework of lifecycle investments and its historical evolution within the field of financial economics. We will take a look at standard heuristics suggested by financial consultancies and compare their welfare outcomes with those of optimal solutions given by theory. We will use the latest theoretical findings by Munk (2016) and show that optimal solutions can be both efficient and easy to comprehend without use of complex dynamic optimization results.
\paragraph*{}Next chapter will review all the relevant literature in this field and show the theoretical developments. Chapter 3 will summarize the theoretical framework and model, we will use in our simulation. Chapter 4 will explain the data sources and the structure of our simulation. Chapter 5 will present the results of the simulation and Chapter 6 will conclude our findings. The used sources will be listed in Bibliography. All the relevant proofs will be available in Appendices. 

\begin{table}
	\centering
	\caption{Largest Turkish Pension Funds}
	\label{table:emekli}
	\begin{tabular}[H]{lc}
		\hline
		Fund name&Fund size\\
		\hline
		AvivaSA Emeklilik ve Hayat&14.8 bln\\
		Anadolu Hayat Emeklilik&14.1 bln\\
		Garanti Emeklilik ve Hayat&11.1 bln\\
		Allianz Yasam ve Emeklilik&10.4 bln\\
		Vakif Emeklilik&6.1 bln\\
		\hline
	\end{tabular}\\
	Source: Pension Monitoring Center (2018)
\end{table}
