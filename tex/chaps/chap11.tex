\chapter{Literature Review} % Main chapter title
\label{litreview} % For referencing the chapter elsewhere, use \ref{introduction} 

The field of financial economics has gone through big changes since its foundation by \citet{markowitz} and \citet{tobin}. They pioneered the mean-variance analysis, which, given some assumptions, suggested that if investors cared about maximizing returns (mean) and minimizing risks (variance), then the optimal ratio of stocks to bonds in a single-period portfolio would be fixed for everyone, the share of former being equal to:

\begin{equation}\label{eq:markowitz}
	\alpha = \frac{\mu - R_f}{\gamma\sigma^2}
\end{equation}

\citet{merton} generalized the problem to multiple periods using dynamic programming and found that it is optimal for all households to repeat the same fixed mean-variance solution every period.

\paragraph{}These results were inconsistent with the popular financial advice suggesting that younger investors should have higher share of stocks in portfolio, and older investors --- higher share of bonds. This advice was summarized by the famous rule of thumb:

\begin{equation}\label{eq:stominus}
	\alpha_t = (100 - t)\%
\end{equation}

Although \citet{samuelson} denied that risk-aversion changes by age, dismissing this advice would question the rationality of investors and ``constitute \textit{prima facie} evidence that people do not optimize'' (\citet{canner}).

\paragraph{}\citet{bodie} solved this problem by adding human capital into the \citet{merton}'s dynamic model and found that for complete markets and constant risk-free labor income, the optimal share of stocks in a portfolio is:

\begin{equation}\label{eq:bodie}
	\alpha_t = \frac{\mu - R_f}{\gamma \sigma^2} \left( \frac{F_t + L_t}{F_t} \right)
\end{equation}

Steady depletion of human capital $L_t$ relative to the financial wealth $F_t$ throughout life, explained the higher share of stocks in younger people. \citet{cgm} extended this idea to the case of variable labor income, to find a recursive solution which could be approximated by the following rule of thumb:

\begin{equation}\label{eq:cgm}
	\alpha_t =
	\begin{cases}
		100\% 			& 	t<40\\
		(200-2.5t)\% 	& 	t\in[40,60]\\
		50\% 			& 	t>60
	\end{cases}
\end{equation}


However, the hump-shaped lifetime stock share graph, observed by \citet{chang} of Federal Reserve, instead of expected downward sloping one, suggested the presence of an opposing force.

\paragraph{}\citet{cocco} found that this force was housing investment, which, due to its large size, crowded out all stocks from younger investors' portfolios. \citet{flavin} supported this view by showing that younger people, who already own the house, tend to invest more aggressively, as was expected by \citet{bodie}.

\paragraph{}\citet{munk} found the same patterns using a series of one-period mean-variance optimizations without any dynamic stochastic modeling tools.

\paragraph{}Finally, \citet{ascheberg} illustrated the existence of long-term cointegration among house prices, stock prices, and labor income, the fact often omitted by previous portfolio researchers for simplicity. In our analysis we will not neglect the correlations as being equal to zero. 
