\chapter{Introduction} % Main chapter title
\label{intro} % For referencing the chapter elsewhere, use \ref{introduction} 


\section{Theory and heuristics of life-cycle investments}
One of the most important investment decisions individuals face in their lives is investment in retirement portfolio. Since the birth of concept of retirement over a century ago, the various advisors have been ubiquitous. They tried to consult people on best ways to invest their money to afford a good standard of living during their old ages. The field of financial economics, however, started analyzing this type of investment decades later, starting from Markowitz's Modern Portfolio Theory (1952). Therefore, this field of theoretical economics has been heavily intertwined with empirical findings of non-academic financial consultants.
\paragraph*{}
In time, financial economists found inefficiencies in portfolio allocations suggested by finacial advisors (Campbell \& Viceira, 2002) and came up with quantitative solutions that would increase investors' welfare. The rapid adoption of Defined Contribution (DC) pension plans, where individuals choose their own pension investment funds and amounts freely, has made it even easier to adopt portfolio decisions described in formulas by economists. At the same time this shifted the whole responsibility on the individuals, and this opened a room for confusion among non-experts, who then decided to naively allocate 50\% of their money to risky assets and the other 50\% to riskless assets.
\paragraph*{} 
To address this issue, lifecycle investment strategies have been introduced by some institutions. They constituted the predefined percentages of risky and riskless fund investments for all ages until retirement. Such portfolios would be in compliance with theory that younger people should invest more in risky assets because this would increase expected earnings and in case of fault, they will be able to reallocate before getting older, and older people should invest more conservatively in less risky assets because they won't have enough time to recover from potential losses. Such "investment menus" were designed to help laymen make their decisions easier while still complying with complex theory. Alas, Turkish consulting firms have not included easy-to-comprehend lifecycle strategies (investment menus) in their bulletins and didn't try to spread transparent information. We will fill this important gap in our paper, but firstly we will recap the state of Turkish pension system.

\section{Turkish pension system}
The main pension funds in Turkey have been public for a long time: three main options existed: SSK for public and private sector workers, ES for civil servants, and Bag-Kur for self-employed workers and farmers. In 2006 they all merged into SGK. Private pensions have gained pace recently. As of January, 2017 a new clause of Turkish Labor Law came into action, that automatically enrolled every wage earner younger than 45 years into "Individual Retirement Scheme" --- a private pension fund. To further incentivize people not to opt out, the government promised to subsidize 25\% of their monthly contributions (as long as this wouldn't exceed 25\% of minimal wage). According to PwC research, this has tremendously increased fund sizes. Under the current system individuals may retire after contributing to a pension fund for at least 10 years and at least reaching the age of 56.
\paragraph*{}
The largest retirement funds in Turkey are listed in the table 1.1. All of them offer 3-4 default investment options with varying degrees of riskiness but they are not lifecycle investment strategies mentioned above. They also provide flexible investment options with ability to change portfolio allocation up to six times a year, but they are not very popular as they assume active involvement in their own portfolio and require a certain level of financial literacy. Not much academic research has been done on Turkish pension systems, and the existing research doesn't provide easy solutions. A recent example of this is Iscanoglu-Cekic's (2016) paper which doesn't consider life cycles and uses dynamic programming in the solution, which is also not accessible to wide audience.

\begin{table}
	\centering
	\caption{Largest Turkish Pension Funds}
	\begin{tabular}[H]{lc}
		\hline
		Fund name&Fund size\\
		\hline
		Anadolu Hayat Emeklilik&8.7 bln\\
		Garanti Emeklilik ve Hayat&7.4 bln\\
		AvivaSA Emeklilik ve Hayat&9.1 bln\\
		Allianz Yasam ve Emeklilik&6.8 bln\\
		Vakif Emeklili&3.5 bln\\
		\hline
	\end{tabular}\\
	Source: Pension Monitoring Center (2016)
\end{table}

\section{Focus of this thesis}
In this thesis we will consider the general framework of lifecycle investments and its historical evolution within the field of financial economics. We will take a look at standard heuristics suggested by financial consultancies and compare their welfare outcomes with those of optimal solutions given by theory. We will use the latest theoretical findings by Munk (2016) and show that optimal solutions can be both efficient and easy to comprehend without use of complex dynamic optimization results.
\paragraph*{}Next chapter will review all the relevant literature in this field and show the theoretical developments. Chapter 3 will summarize the theoretical framework and model we will use in our simulation. Chapter 4 will explain the data sources and the structure of our simulation. Chapter 5 will present the results of the simulation and Chapter 6 will conclude our findings. The used sources will be listed in References chapter. All the relevant proofs will be available in Appendices. 
