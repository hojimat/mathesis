\chapter{Conclusion}
\label{conclusion}

In this thesis, we have reviewed the concept of "life-cycle investment" and summarized the common models and heuristics. We have presented our model as an application of Munk's (2016) recent findings and Olear's (2016) simulation techniques into Turkish retirement market.

\paragraph{}We have collected historical data to calibrate and estimate the best parameters to be used in our simulation. Using these parameters, we have constructed heterogeneous agents, who worked and invested throughout their lifetime. We considered different investment models that our hypothetical agents would use and calculated the resulting investment capitals. Finally, we have calculated and compared the welfare effects of all popular models to the individualized Munk's solutions.

\paragraph{}We have concluded that heuristic, provided by \citet{cocco} based on their underlying dynamic programming model, provides a higher utility on average than any other investment option.

\paragraph{}We also found that even naive life-cycle investment heuristics perform better than fixed Markowitz solution. 

\paragraph{}Our simulation showed that for risk-averse individuals with riskless wages, it is best to invest in housing in a proportion proposed by Munk.

\paragraph{}We propose these models to Turkish pension providers, as these options will increase their retirement welfare considerably.
