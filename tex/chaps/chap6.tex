\chapter{Conclusion}
\label{conclusion}

In this thesis we have reviewed the current state of pension investments in Turkey and the history and recent developments of a field of financial economics. We have reviewed the concept of "lifecycle investment" and summarized the common models and heuristics.

\paragraph{}We have presented our model as an application of Munk's (2016) recent findings and Olear's (2016) simulation techniques into Turkish retirement market. We have collected historical data to calibrate and estimate the best parameters to be used in our simulation.

\paragraph{}Using these parameters, we have constructed heterogeneous agents, who worked and invested throughout their lifetime. We considered different investment models that our hypothetical agents would use and calculated the resulting investment capitals.

\paragraph{}Finally, we have calculated and compared the welfare effects of all popular models to the individualized Munk's solutions. We have concluded that for rich-to-middle class citizens, the individiualized strategies considerably increase their welfare. Moreover, the solutions we proposed can be solved analytically without complex dynamic optimizations, and therefore are easy to interpret to households. We also found that even naive lifecycle investments perform better than fixed lifetime investment.

\paragraph{}We propose these models to Turkish pension providers and to Turkish working-age households belonging to middle-to-upper class, as these options will increase their retirement welfare considerably.
