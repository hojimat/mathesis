\chapter{Model}
\label{model}

\section{Labor income process}

We model the labor income process as a function of individiual characteristics plus aggregate and idiosyncratic shocks for working people, and as the percentage $\lambda$ of the last received wage for the retired. This is summarized in Cocco et al. (2005) as follows:

\begin{center}
	$\log(Y_{it}) =
		\begin{cases}
			f(t,Z_{it}) + v_{it} + \epsilon_{it}, & t \leq T \\
			log(\lambda) + f(T, Z_{iT}) + v_{iT}, & t > T
		\end{cases}
	$
\end{center}

where $T$ is the retirement age and $f(t, Z_{it})$ is the log-wage regression outcome for individual $i$ at time $t$. The error terms are decomposed as:

\begin{center}
	$v_{it} = v_{i,t-1} + u_{it}$,\\
	$u_{it} = \xi_t + \omega_{it}$
\end{center}

and distributed as:

\begin{center}
	$u_{i} \sim N(0, \sigma^2_u)$,\\
	$\xi \sim N(0,\sigma^2_{\xi})$,\\
	$\omega_{i} \sim N(0, \sigma^2_{\omega})$.
\end{center}

We use Olear's (2016) approach (Appendix B) to transform the main equation into the following:

\begin{center}
	$Y_{i,t+1} = 
	\begin{cases}
		Y_{it} (1 + g_{i,t+1} + \xi_t + \omega_{it}), & t \leq T \\
		\lambda (1 + f(T, Z_{iT}) + v_{iT}), & t > T
	\end{cases}	
	$
\end{center}

where $g_{i,t+1} = f(t+1, Z_{i,t+1}) - f(t, Z_{it})$, $\xi$ is the aggregate shock and $\omega_{i}$ is idiosyncratic shock.


\subsection{Correlations}

To derive the above equation, we must construct the aggregate labor income shock $\xi$. Following the Approach of Ascheberg et al. (2013) we want labor income series to be correlated with both stock series and housing series. To do that we first create three uncorrelated standard normally distributed random series $\epsilon_{st}$, $\epsilon_{ht}$, $\epsilon_{yt}$ and multiply them by the Cholesky decomposition $Q$ of the correlation matrix $R$, i.e. $R = QQ'$, where:

\begin{center}
	$R = \begin{bmatrix}
					1 & \rho_{sh} & \rho_{sy} \\
					\rho_{hs} & 1 & \rho_{hy} \\
					\rho_{ys} & \rho_{yh} & 1
			\end{bmatrix}
	$
\end{center}

Stock, housing and labor income series then take the form of expected rate of return plus the volatility multiplied by the modified error terms. For details please refer to Appendix A:\\
$\frac{\Delta S_{t+1}}{S_t} = \mu_s + \sigma_s \cdot \epsilon_{st}$\\
$\frac{\Delta H_{t+1}}{H_t} = \mu_h + \sigma_h \cdot \left(\rho_{hs}\epsilon_{st} + (\sqrt{1-\rho^2_{hs}})\epsilon_{ht}\right)$\\
$\frac{\Delta Y_{t+1}}{Y_t} = \mu_v + \sigma_v \cdot \left(\rho_{ys}\epsilon_{st} + \left(\frac{\rho_{yh} - \rho_{sh}\rho_{sy}}{\sqrt{1-\rho^2_{sh}}}\right)\epsilon_{ht} + \left(\sqrt{1-\rho^2_{ys}-(\frac{\rho_{yh} - \rho_{sh}\rho_{sy}}{\sqrt{1-\rho^2_{sh}}})^2}\right)\epsilon_{vt}\right)$

\section{Welfare measurement}
Similarly to Cocco et al. (2005) we use CRRA utility function in our model:

\begin{center}
	$E_1[U(c)] = E_1 \left[\displaystyle\sum^T_{t=1} \delta^{t-1} \displaystyle\prod^{t-1}_{j=0} p_j \cdot \frac{c^{1-\gamma}_{it}}{1-\gamma}\right]$
\end{center}

where $p_k$ is the probability of survival from time $k-1$ to time $k$. Note that we omitted the bequest motives from the original formulation, thus retired person consumes all of his income at any given time.

%\paragraph*{}Following Olear (2016) we use certainty equivalent consumptions (CEC) instead of expected utilities to compare the welfare effects between different lifecycle choices. Appendix C shows the calculation of the following formula for CEC:

%\begin{center}
%	$CEC = \left( \frac{E[U(c)]\cdot(1-\gamma)}{\sum^T_{t=1} \delta^{t-1} \prod^{t-1}_{j=0} p_j} \right)^{1/(1-\gamma)}$
%\end{center}


\section{Individualization}

To derive the individual lifecycle portfolios for every wealth type we use a special case of Munk (2016) with housing investment and human capital included. The similar approach and its solution has been done by Olear (2016), but she included housing capital as a pre-existing wealth bought for an outside mortgage, independent of the investment decision. In this paper, we consider housing as a purely financial investment, in line with Munk (2016). First, note that total wealth at any time consists of financial, housing and human capital: $W_t = F_t + L_t$. Financial wealth corresponds to actual financial and housing asset holdings at time $t$, and human capital corresponds to the present value of all the future wages until retirement. This gives the mean:

\begin{center}
	$E[\frac{W_1}{W_0}] = \frac{F_1 + L_1}{W_0} =  \frac{F_0}{W_0} (1 + r_p) + \frac{L_0}{W_0}(1+\mu_L)$
\end{center}

where $r_p = r_f(1-\pi - \pi_h) + \pi \cdot \mu_s + \pi_h \cdot \mu_h$ is a portfolio return, $\pi$ and $\pi_h$ are shares of total wealth invested in stocks and housing respectively, $r_f$ is a rate of return on a risk-free asset (bond), and $\mu_s = E[r]$ and $\mu_h$ are expected stock and housing returns respectively. The volatility is given by:

\begin{center}
	$var(\frac{W_1}{W_0}) = (\frac{F_0}{W_0})^2 \boldsymbol{\pi} \Sigma \boldsymbol{\pi} + (\frac{L_0}{W_0})^2(\sigma^2_L) + 2 \cdot \frac{F_0 L_0}{W^2_0} \cdot \pi\cdot cov(\boldsymbol{r},\mu_L)$
\end{center}

where: $\boldsymbol{\pi} = \left( \pi, \pi_h \right)$ and $\boldsymbol{r} = \left( \mu_s, \mu_h \right)$. Substituting this into the unconstrained mean-variance maximization and solving yields:

\begin{center}
	$\pi_{t+1} = \frac{\mu_s - r_f}{\gamma \sigma^2_s}  + \frac{L_t}{F_t} \cdot \left(\frac{\mu_s - r_f}{\gamma \sigma^2_s} - \frac{\rho_{SL}\sigma_L}{\sigma_S} \right)$
\end{center}

without housing investment, and:

\begin{center}
	$\pi_{t+1} = \frac{1}{\gamma (1 - \rho^2_{SH}) \sigma_S} \cdot \frac{W_t}{F_t} \left( \frac{\mu_s - r_f}{\sigma_S} - \rho_{SH} \frac{\mu_h - r_f}{\sigma_h} \right) - \frac{L_t}{F_t} \cdot \frac{\sigma_L}{\sigma_S} \frac{\rho_{SL} - \rho_{SH}\rho_{HL}}{1 - \rho^2_{SH}}$\\
	$\pi_{h,t+1} = \frac{1}{\gamma (1 - \rho^2_{SH}) \sigma_H} \cdot \frac{W_t}{F_t} \left( \frac{\mu_h - r_f}{\sigma_h} - \rho_{SH} \frac{\mu_s - r_f}{\sigma_s} \right) - \frac{L_t}{F_t} \cdot \frac{\sigma_L}{\sigma_h} \frac{\rho_{HL} - \rho_{SH}\rho_{SL}}{1 - \rho^2_{SH}}$
\end{center}

when housing is considered as a second risky financial investment. In line with Munk (2016), we calculate the risk free asset share as $\left( 1 - \pi - \pi_h \right)$ and only then impose constraints. If any of the above shares is negative, we equate it to zero, and if the sum of remaining shares exceeds $1$, we divde all of the shares by their sum to obtain a proportionate asset allocation. 

\section{Retirement income}

The funds invested in retirement are modeled to be paid back in annuities. Thus, we ignore the case when some share of matured fund can be withdrawn immediately. Further, in order for our individualization analysis (based on house posessions) to have an effect, we also use reverse mortgages proposed. In theory, reverse mortgages pay retired individuals fixed annuity in return for inheriting the house after she dies. This is a plausible tool, because we ignore all bequest motives, and it liquidifies the housing posessions, although it is not yet available for Turkish investors. 

\paragraph{}So, at the age of 57, the price of owned house is calculated and is added to the matured pension amount ($MP$) to obtain total wealth:

\begin{center}
	$W_{57} = H_{57} + MP$
\end{center}

All of the $W_{57}$ is used to buy an annuity which will repay an individual $\frac{W_{57}}{1+\sum^{100}_{t=58} \frac{p_t}{1+r_f}}$ annually. The calculation details are available in Appendix D. 
