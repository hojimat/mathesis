\chapter{Literature Review}
\label{litreview}

\section{Beginnings of financial economnics}

\paragraph*{}The financial economics is generally thought to be started with Modern Portfolio Theory (MPT) by Markowitz(1952). He pioneered the mean-variance analysis and was followed by Mutual Funds Separation Theorem of Tobin(1958). The premise of the model was, that if investors care only about the return and the volatility (modelled by mean and variance(or standard deviation) respectively) over a single period, then there is a straight line representing a fixed ratio of risky assets in the optimal portfolio. We summarize their model as follows:

\paragraph*{Mean-Variance Analysis}
Let there be two assets, risky and risk-free with returns $R$ and $R_f$ respectively. Let $\alpha$ be the ratio of total wealth invested in a risky asset. Then the portfolio return is:

\begin{center}
  $R_p = \alpha E[R] + (1-\alpha) R_f$
\end{center}

We want to choose $\alpha$ that maximizes the expected return and minimizes the volatility of the portfolio. Markowitz solves the following unconstrained optimization problem:

\begin{center}
  $\displaystyle\max_{\alpha} \{ E[R_p] - \frac{\gamma}{2}\sigma^2_p \}$
\end{center}

where $\gamma$ is risk-aversion coefficient. The classical solution is:

\begin{center}
	$\alpha = \frac{E[R] - R_f}{\gamma\sigma^2}$
\end{center}

\paragraph*{}This is a crucial result used a lot in industry, academia and MBA courses but most importantly, revived recently by Munk(2016) which we will explore later in this chapter. The issue with this basic model was that it demanded fixed risky asset ratio for everybody and thus could not explain why (i) younger investors take more risks than older ones and (ii) aggressive investors invest more in stocks than bonds compared to the conservative ones. The model also didn't include loss aversion of people, because although this solution performs best on average in the long run, when it underperforms, it does so in a high way and investors are willing to sacrifice possible gains for loss avoidance.

 \paragraph*{}Following Markowitz, Merton(1970) and Samuelson(1969) introduced a framework to understand long term portfolio investments using changes of investment opportunities during the life. But Merton used dynamic programming and didn't have closed form solutions for the general case.

  also this implies the myopic choice, meaning whenever the financial wealth is the only wealth, the time horizon is not important for them (assumig that their relative risk aversion is does not depend on wealth). Again, they ignore such claims that younger should risk more, because they have more time.

   
\section{Advancements in thought}
\item Bodie, Campbell, Samuelson introduced human capital to the problem.
  They view labor income as a divident on individual's human wealth. It is non-tradeable because of moral hazard problem. Thus the claims of immediate money for claims of future works are not legally binding, because human rights forbid any such contracts to prevent all forms of slavery.
  This was an important paper and could be summarized as follows:
  
\item someone introduced housing
\item Munk introduced closed form solution
\section{Individualized portfolios}
\item Olear did for Netherlands
\item someone did for other countries
