\chapter*{Abstract\\\ttitle}

We perform an in-depth welfare comparison of the most common life-cycle investment strategies provided by retirement funds or suggested by classical portfolio theory in the case of households in Turkey. To perform our benchmarking, we construct heterogeneous agents who work and invest throughout their lifetime, using parameters calibrated from the historical data. We find that to households with upper-to-middle income, individually customized portfolios result in considerable welfare gains, while ``off-the-shelf'' life-cycle portfolio allocations perform better for households with lower income. We also show that life-cycle investment options outperform ``fixed over the lifetime'' options. Finally, we find that risk-averse individuals with volatile wages, can maximize their welfare by investing in housing as suggested by \citet{munk}.
\newpage

\chapter*{Özet\\\ttitletr}
 
Bu çalışmamızda, klasik portföy teorisinin ve endüstri uzmanların sunduğu yaşamboyu yatırım stratejilerin refah seviyelerini kıyaslayacağız. Kıyaslamayı gerçekleştirmek için, tarihi verilerden belirlediğimiz parametrelere göre, iş hayatı boyunca yatıran heterojen ajanları kuracağız. Düşük maaşlı ajanlar için klasik modellerin daha karlı olduğunu ve orta-ve-üst maaşlı ajanlar için kişiselleştirilmiş modellerin daha karlı olduğunu öğreneceğiz. \citet{munk}'un sunduğu modeli kullanarak, riskten kaçınan ajanların, ev yatırımını yaparak, refah seviyelerini maksimize edebileceğini göreceğiz.
