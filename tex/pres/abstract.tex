\chapter*{Abstract}
\ttitle

\vspace{20pt}
We review the current state of Turkish pension system and the history and developments of financial economics. We apply Munk's (2016) individualized lifecycle investment model to simulate Turkish retirement process, and compare the welfare effects with default retirement portfolio options provided by retirement funds or suggested by classical portfolio theory. We find that for upper-to-middle class citizens, individualizing portfolios in Munk's sense results in considerable welfare increases. 

\newpage

\chapter*{Özet}

\ttitletr

\vspace{20pt}
 
Bu tezde, önce Türkiye'deki emeklilik sistemini ve finansal ekonominin tarihçesini ve gelişmelelerini özetledik. Munk'un (2016) kişiselleştirilmiş yaşamboyu yatırım modelini, Türkiye'deki emeklilik sürecini simule etmeye, ve gerçek emeklilik fonların veya klasik portföy teorisinin tavsiye ettiği yatırım opsiyonlarıyla kıyaslamaya kullandık. Üst ve orta sınıf yatırımcıların, bu kişiselleştirilmiş yatırımları kullandığı takdirde, refah seviyesinin artacağını gösterdik. 
